\documentclass[14pt]{extarticle} 
\usepackage{amsmath,mathtools,amsfonts,amsthm,amssymb,hyperref}
\usepackage{wasysym,geometry,bussproofs,latexsym,parskip,bookmark}
\usepackage{mathtools,float}
\newtheorem{defn}{Definition}
\newtheorem{thm}{Theorem}
\newtheorem{claim}{Claim}
\newtheorem{lemma}{Lemma}
\newcommand{\dps}{\displaystyle}
\hypersetup{colorlinks,allcolors=blue,linktoc=all}
\geometry{a4paper} 
\geometry{margin=0.5in}
\title{Math for CS 2015/2019 solutions to ``In-Class Problems Week 10, Mon. (Session 24)''}
\author{https://github.com/spamegg1}
\begin{document}
\maketitle
\tableofcontents

\section{Problem 1}
Recall that for functions $f, g$ on $\mathbb{N}$, $f = O(g)$ iff

$$
\exists c \in \mathbb{N} \,\, \exists n_0 \in \mathbb{N} \,\, \forall n \geq n_0 \,\, ( c \cdot g(n) \geq |f(n)|) \,\,\,\,\,\,\,\,\,\,(1)
$$

For each pair of functions below, determine whether $f = O(g)$ and whether $g = O(f)$. In cases where one function is $O()$ of the other, indicate the {\it smallest nonnegative integer, c,} and for that smallest $c$, the {\it smallest corresponding nonnegative integer} $n_0$ ensuring that condition (1) applies.
\subsection{(a)}
$f(n) = n^2, g(n) = 3n$
\begin{proof}
In this case $f \neq O(g)$ (because $n^2$ dominates $n$), and $g = O(f)$, with $c = 1$, $n_0 = 3$.
\end{proof}

\subsection{(b)}
$f(n) = (3n-7)/(n+4), g(n) = 4$
\begin{proof}
In this case $f = O(g)$ with $c = 1$, $n_0 = 0$ because $|f(n)| < 3$ for all $n \in \mathbb{N}$.

Also $g = O(f)$ with $c = 2, n_0 = 15$. Since $\dps\lim_{n\to\infty} f(n) = 3$, the smallest possible $c$ is 2. For $c = 2$, the smallest possible $n_0 = 15$ which follows from the requirement that $2 f(n_0) \geq 4$.
\end{proof}

\subsection{(c)}
$f(n) = 1+(n\sin(n\pi / 2))^2, g(n) = 3n$
\begin{proof}
Notice $g = \Theta(n)$.

$f \neq O(g)$ because $f(2n+1) = n^2+1$ for all $n \in \mathbb{N}$ but $g = \Theta(n)$. $n^2$ cannot be bounded by a constant multiple of $n$.

$g \neq O(f)$ because $f(2n) = 1$ for all $n \in \mathbb{N}$ but $g = \Theta(n)$. $n$ cannot be bounded by a constant.
\end{proof}

\section{Problem 2}
\subsection{(a)}
Indicate which of the following asymptotic relations below on the set of nonnegative real-valued functions are equivalence relations (E), strict partial orders (S), weak partial orders (W), or none of the above (N).

\subsubsection{}
$f \sim g$, the ``asymptotically equal'' relation.

{\bf Solution.} (E)

\begin{proof}
$f \sim f$ because $\dps\lim_{x\to\infty}f(x)/f(x) = 1$. So the relation is reflexive.

If $f \sim g$ then $\dps\lim_{x\to\infty}f(x)/g(x) = 1$. Then $\dps\lim_{x\to\infty}g(x)/f(x) = 1$. So the relation is symmetric.

If $f \sim g$ and $g \sim h$ then $\dps\lim_{x\to\infty}f(x)/g(x) = 1$ and $\dps\lim_{x\to\infty}g(x)/h(x) = 1$. Then $\dps\lim_{x\to\infty}f(x)/h(x) =\lim_{x\to\infty} (f(x)/g(x))/(g(x)/h(x)) = \lim_{x\to\infty} (f(x)/g(x)) \cdot \lim_{x\to\infty} (g(x)/h(x)) = 1/1 = 1$. So the relation is transitive.
\end{proof}

\subsubsection{}
$f = o(g)$, the ``little Oh'' relation.

{\bf Solution.} (S)

\begin{proof}
For any $f$, we have $\dps\lim_{x\to\infty}f(x)/f(x) = 1 \neq 0$, hence $f \neq o(f)$, so the relation is irreflexive.

If $f = o(g)$ and $g = o(h)$ then $\dps\lim_{x\to\infty}f(x)/g(x) = 0$ and $\dps\lim_{x\to\infty}g(x)/h(x) = 0$.

Assume $\epsilon > 0$. There exist $x_1$ and $x_2$ such that

for all $x > x_1$ we have $\dps\frac{f(x)}{g(x)} < \sqrt{\epsilon}$, and for all $x > x_2$ we have $\dps\frac{g(x)}{h(x)} < \sqrt{\epsilon}$. 

Then for all $x > \max(x_1, x_2)$ we have $\dps \frac{f(x)}{h(x)} = \frac{f(x)}{g(x)} \cdot \frac{g(x)}{h(x)} < \sqrt{\epsilon} \cdot \sqrt{\epsilon} = \epsilon$.

So $\dps\lim_{x\to\infty}f(x)/h(x) = 0$. So the relation is transitive.
\end{proof}

\subsubsection{}
$f = O(g)$, the ``big Oh'' relation.

{\bf Solution.} (N)

\begin{proof}
The relation is reflexive: $f = O(f)$ (just take $C = 1$).

The relation is transitive: assume $f = O(g)$ and $g = O(h)$. There exist $x_0, x_1$ and constants $C, D$ such that $f(x) \leq Cg(x)$ for all $x \geq x_0$, and $g(x) \leq Dh(x)$ for all $x \geq x_1$. Let $x_2 = \max(x_0, x_1)$ and $E = CD$. Then for all $x \geq x_2$ we have $f(x) \leq Eh(x)$ so $f = O(h)$.

The relation is not symmetric: $x = O(x^2)$ but $x^2 \neq O(x)$.

The relation is not asymmetric: $f = O(f)$ is true but it does not imply NOT$(f = O(f))$ (which is false).

The relation is not antisymmetric: let $f(x) = x$ and $g(x) = 2x$ so $f \neq g$. We have $f = O(g)$ but this does not imply NOT$(g \neq O(f))$, because $g = O(f)$ is also true.
\end{proof}

\subsubsection{}
$f = \Theta(g)$, the ``Theta'' relation.

{\bf Solution.} (E)

\begin{proof}
$f = \Theta(f)$ because $f = O(f)$ (take $C = 1$). So the relation is reflexive.

Assume $f = \Theta(g)$. Then by definition $f = O(g)$ and $g = O(f)$. This implies $g = \Theta(f)$. So the relation is symmetric.

Assume $f = \Theta(g)$ and $g = \Theta(h)$. Then by definition $f = O(g)$ and $g = O(f)$, and $g = O(h)$ and $h = O(g)$. Since big-O is transitive (as shown above), $f = O(g)$ and $g = O(h)$ imply $f = O(h)$. Similarly $h = O(g)$ and $g = O(f)$ imply $h = O(f)$. So $f = \Theta(h)$. Therefore the relation is transitive.
\end{proof}

\subsubsection{}
$f = O(g)$ AND NOT$(g = O(f))$.

{\bf Solution.} (S)

\begin{proof}
This relation is clearly irreflexive: $f = O(f)$ AND NOT$(f = O(f))$ is not possible.

It is also transitive: assume $f = O(g)$ AND NOT$(g = O(f))$, and $g = O(h)$ AND NOT$(h = O(g))$. Because big-O is transitive (shown above) we have $f = O(h)$. We need to show NOT$(h = O(f))$. Argue by contradiction and assume $h = O(f)$. Since $f = O(g)$, by transitivity $h = O(g)$ which contradicts NOT$(h = O(g))$. So our assumption was false, and NOT$(h = O(f))$ is true.
\end{proof}

\subsection{(b)}
Indicate the implications among the assertions in part (a). For example, $f = o(g)$ IMPLIES $f = O(g)$.
\begin{proof}
There are 5 assertions: \\
(1) $f \sim g$,\\ (2) $f = o(g)$,\\ (3) $f = O(g)$,\\ (4) $f = \Theta(g)$,\\ (5) $f = O(g)$ AND NOT$(g = O(f))$.

There is no relation between (1) and (2): (1) says that a limit is 1, (2) says the same limit is 0.

By Lemma 13.7.7, (1) implies (3). 

(1) implies (4): if $\lim f/g = 1$ then $f = O(g)$ and $g = O(f)$ (just take $C = 1$), so $f = \Theta(g)$. However (4) does not imply (1): again we can use the above example, we see $x = \Theta(2x)$ but $1/2 \neq 1$ so $x \not\sim 2x$.

There is no relation between (1) and (5): $f \sim g$ implies both $f = O(g)$ and $g = O(f)$.

By Lemma 13.7.7, (2) implies (3). But (3) does not imply (2): for example $f(x) = x$, $g(x) = 2x$ so $C=2$ shows $f = O(g)$ but $\lim f/g = 1/2 \neq 0$, therefore $f \neq o(g)$. Same example shows (3) does not imply (1) because $1/2 \neq 1$.

There is no relation between (2) and (4): we have $x = o(x^2)$ but $x \neq \Theta(x^2)$. Again similar to above, $x = \Theta(2x)$ but $x \neq o(2x)$ because $1/2 \neq 0$.

(2) and (5) are equivalent: $\lim f/g = 0$ iff $\lim f/g$ is finite and $\lim g/f = infty$, iff $f = O(g)$ and NOT$(g = O(f))$.

(4) clearly implies (3) by definition of big-Theta, but (3) does not imply (4): we have $x = O(x^2)$ but $x \neq \Theta(x^2)$.

(5) clearly implies (3) but not vice versa.

There is no relation between (5) and (4), because $f = \Theta(g)$ iff $f = O(g)$ AND $g = O(f)$.

We can summarize as follows:
$$
\begin{array}{rcc}
1&\longrightarrow&4\\
&\searrow&\downarrow\\
(2 \leftrightarrow 5)&\longrightarrow&3\\
\end{array}
$$
\end{proof}

\section{Problem 3}
{\bf False Claim.} $2^n = O(1)$. \,\,\,\,\,\,(2)

Explain why the claim is false. Then identify and explain the mistake in the following bogus proof.

{\it Bogus proof.} The proof is by induction on $n$ where the induction hypothesis, $P(n)$, is the assertion (2).

{\bf base case:} $P(0)$ holds trivially.

{\bf inductive step:} We may assume $P(n)$, so there is a constant $c > 0$ such that $2^n \leq c \cdot 1$. Therefore,
$$
2^{n+1} = 2 \cdot 2^n \leq (2c) \cdot 1
$$
which implies that $2^{n+1} = O(1)$. That is, $P(n + 1)$ holds, which completes the proof of the inductive step. 

We conclude by induction that $2^{n} = O(1)$ for all $n$. That is, the exponential function is bounded by a constant.

\begin{proof}
A function is $O(1)$ iff it is bounded by a constant, and since the function $2^n$ grows unboundedly with $n$, it is not $O(1)$.

The mistake in the bogus proof is in its misinterpretation of the expression $2^n$ in assertion (2). The intended interpration of (2) is

\begin{center}
Let $f$ be the function defined by the rule $f(n) \Coloneqq 2^n$. Then $f = O(1)$. \,\,\,\,\,\,\,\,(3)
\end{center}

But the bogus proof treats (2) as an assertion, $P(n)$, about $n$. Namely, it misinterprets (2) as mean­ing:
\begin{center}
Let $f_n$ be the constant function equal to $2^n$. That is, $f_n(k) \Coloneqq 2^n$ for all $k \in \mathbb{N}$. Then 
$$
f_n = O(1)\,\,\,\,\,\,\,\,\,\,(4)
$$
\end{center}

Now (4) is true since every constant function is $O(1)$, and the bogus proof is an unnecessarily complicated, but correct, proof that that for each $n$, the constant function $f_n$ is $O(1)$. But in the last line, the bogus proof switches from the misinterpretation (4) and claims to have proved (3).

So you could say that the exact place where the proof goes wrong is in its first line, where it defines $P(n)$ based on misinterpretation (4). Alternatively, you could say that the proof was a correct proof (of the misinterpretation), and its first mistake was in its last line, when it switches from the misinterpretation to the proper interpretation (3).
\end{proof}

\section{Problem 4 (Supplemental problem)}
Assign true or false for each statement and prove it.

\subsection{(a)} $n^2 \sim n^2 + n$
\begin{proof}
True:
$$
\lim_{n\to\infty} \frac{n^2}{n^2+n} = \lim_{n\to\infty} \frac{1}{1+\frac{1}{n}} = \frac{1}{1+0} = 1
$$
\end{proof}

\subsection{(b)} $3^n = O(2^n)$
\begin{proof}
False. For every $n_0$ and every positive constant $C$ we can find $n \geq n_0$ such that $3^n \geq C \cdot 2^n$. For example, we can let $n = \max(n_0, \lceil\frac{\log C}{\log (3/2)}\rceil + 1)$.
\end{proof}

\subsection{(c)} $n^{\sin(n\pi/2) + 1} = o(n^2)$
\begin{proof}
False. 

The limit $\dps\lim_{n \to \infty} \frac{n^{\sin(n\pi/2) + 1}}{n^2}$ does not exist, because $n^{\sin(n\pi/2) + 1}$ oscillates between $n, n^2$ and $-n^2$; so $\dps\frac{n^{\sin(n\pi/2) + 1}}{n^2}$ oscillates between $0, 1, -1$.
\end{proof}

\subsection{(d)} $\dps n = \Theta\left(\frac{3n^3}{(n+1)(n-1)}\right)$
\begin{proof}
True. $\dps n = O\left(\frac{3n^3}{(n+1)(n-1)}\right)$ by choosing $C = 1/3$, and $\dps\frac{3n^3}{(n+1)(n-1)}=O(n)$ by choosing $C = 3$.
\end{proof}

\section{Problem 5 (Supplemental problem)}
Give an elementary proof (without appealing to Stirling’s formula) that $\log(n!) = \Theta(n \log n)$.
\begin{proof}
One elementary proof goes as follows: First,
$$
\log(n!) = \sum_{i=1}^n \log(i) < \sum_{i=1}^n \log(n) = n\log(n)
$$
On the other hand,
$$
\begin{array}{rcl}
\log(n!) &=&\dps \sum_{i=1}^n \log(i) \\
&>&\dps\sum_{i=\lceil(n+1)/2\rceil}^n \log(i)\\
&>&\dps\sum_{i=\lceil(n+1)/2\rceil}^n \log(n/2)\\
&>&\dps\frac{n}{2}\cdot\log(n/2)\\
&=&\dps\frac{n\log(n)}{2} - \frac{n}{2}\\

&>&\dps\frac{n\log(n)}{2} - \frac{n\log(n)}{6}\,\,\,\,\,\,\,\,\,\,\text{(for } n>8)\\
&=&\dps n\log(n)/3\\
\end{array}
$$

Therefore $(1/3)n\log(n) < \log(n!) < n\log(n)$ for $n > 8$, proving $\log(n!) = \Theta(n \log n)$.
\end{proof}
\end{document}
