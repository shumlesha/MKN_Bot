\documentclass[14pt]{extarticle} 
\usepackage{amsmath,mathtools,amsfonts,amsthm,amssymb,hyperref}
\usepackage{wasysym,geometry,bussproofs,latexsym,parskip,bookmark}
\usepackage{mathtools}
\newtheorem{defn}{Definition}
\newtheorem{thm}{Theorem}
\newtheorem{claim}{Claim}
\newtheorem{lemma}{Lemma}
\hypersetup{colorlinks,allcolors=blue,linktoc=all}
\geometry{a4paper} 
\geometry{margin=0.5in}
\title{Math for CS 2015/2019 solutions to ``In-Class Problems Week 1, Fri. (Session 2)''}
\author{https://github.com/spamegg1}
\begin{document}
\maketitle
\tableofcontents

\section{Problem 1}

Prove that if $a \cdot  b = n$, then either $a$ or $b$ must be $\leq n$, where $a$, $b$, and $n$ are nonnegative real numbers. Hint: by contradiction, Section 1.8 in the course textbook.

\begin{proof}
1. Assume $a, b, n$ are nonnegative real numbers, and $a \cdot b = n$. 

2. Argue by contradiction. Assume $a > \sqrt{n}$ and $b > \sqrt{n}$.

3. Since all the numbers involved $a, b, n, \sqrt{n}$ are nonnegative, we can multiply the two inequalities in (2) to get: $a \cdot b > \sqrt{n} \cdot \sqrt{n}$.

4. Using (1) we can replace $a \cdot b$ with $n$, so (3) gives us: $n > n$, a contradiction.

5. Our assumption in (2) must be false, therefore either $a \leq \sqrt{n}$ or $b \leq \sqrt{n}$.
\end{proof}

\section{Problem 2}

Generalize the proof of Theorem 1.8.1 repeated below that $\sqrt{2}$ is irrational in the course textbook. For example, how about $\sqrt{3}$?

We want to prove:

\begin{thm}
$\sqrt{3}$ is an irrational number.
\end{thm}

First we will need another result:

\begin{lemma}
Assume $n$ is a positive integer. If $3$ divides $n^2$, then $3$ divides $n$.
\end{lemma}

\textbf{This is actually Problem 1.10 part (b) in the textbook! Prof. Meyer tells us to do it in the proof of $\sqrt{2}$ is irrational.}

\begin{proof} 

1. Assume $n$ is a positive integer and $3$ divides $n^2$. 

2. By definition of divisibility there exists an integer $k$ such that $3k = n^2$ (we will need this later below).

3. Argue by contradiction and assume that $3$ does not divide $n$.

4. By the Quotient-Remainder Theorem there exist integers $q$, $r$ such that $n = 3q + r$ where $0 \leq r < 3$.

5. Since $3$ does not divide $n$, $r$ cannot be $0$. So $r$ must be $1$ or $2$.

6. \textbf{Case 1}. $r = 1$.

6.1. Then $n = 3q + 1$. So $n^2 = (3q+1)^2 = 9q^2 + 6q + 1$.

6.2. So $3k = 9q^2 + 6q + 1$, dividing by 3 we get $k = 3q^2 + 2q + \frac{1}{3}$. 

6.3. Moving terms, we get $k - 3q^2 - 2q = \frac{1}{3}$. This is a contradiction! Because the left-hand side $k - 3q^2 - 2q$ is an integer, but the right-hand side $\frac{1}{3}$ is not an integer.

7. \textbf{Case 2}. $r = 2$.

7.1. Then $n = 3q + 2$. So $n^2 = (3q+2)^2 = 9q^2 + 12q + 4$.

7.2. So $3k = 9q^2 + 12q + 4$, dividing by 3 we get $k = 3q^2 + 4q + \frac{4}{3}$. 

7.3. Moving terms, we get $k - 3q^2 - 4q = \frac{4}{3}$. This is a contradiction! Because the left-hand side $k - 3q^2 - 4q$ is an integer, but the right-hand side $\frac{4}{3}$ is not an integer.

8. The two cases in (6) and (7) are exhaustive of all possibilities, and in all cases we had a contradiction.

9. Therefore our assumption must have been false, so $3$ divides $n$.
\end{proof}

Now we can begin the proof of the Theorem.

\begin{proof}
1. Argue by contradiction and assume $\sqrt{3}$ is rational.

2. By the definition of a rational number, there exist integers $n$ and $d$ such that $\sqrt{3} = \frac{n}{d}$ where $d \neq 0$ and $n$ and $d$ have no common divisors greater than 1. 

Without loss of generality, we may assume that both $n$ and $d$ are both positive, since $\sqrt{3}$ is positive.

3. Squaring both sides we get $3 = \frac{n^2}{d^2}$.

4. Multiplying both sides by $d^2$ we get $3d^2 = n^2$.

5. From this equation we notice that $3$ divides $n^2$. (Because there exists an integer $k = d^2$ such that $n^2 = 3k$, which is the definition of divisibility). 

6. By (5) and the Lemma, $3$ divides $n$.

7. By (6) and the definition of divisibility, there exists an integer $m$ such that $3m = n$.

8. Substituting (7) into (4) we get $3d^2 = (3m)^2 = 9m^2$.

9. Dividing by 3, we get $d^2 = 3m^2$. This means $d$ is divisible by 3, which is a contradiction to the fact that $n$ and $d$ have no common divisors greater than 1.

10. Therefore our initial assumption was false, hence $\sqrt{3}$ is irrational.
\end{proof}

\subsection{Generalizing even further}

This subsection is fairly hard and is optional.

How far can this Theorem be generalized? Is $\sqrt{4}$ irrational too? No, it's equal to 2. Where would the proof go wrong if we tried it on $\sqrt{4}$?

Let $m$ vary over the positive integers, and consider the general statement: ``$\sqrt{m}$ is irrational.'' Intuitively, it seems like this should be true as long as $m$ itself is not a perfect square. If we go through the proof, we end up with a step where $md^2 = n^2$, and we notice $m$ divides $n^2$. Then we would have to prove the Lemma, that is, if $m$ divides $n^2$ then $m$ divides $n$, and derive the contradiction similarly.

So, is it true that if $m$ and $n$ are positive integers, \textbf{$m$ is not a perfect square}, and $m$ divides $n^2$, then $m$ divides $n$? Not quite. We can let $n = pq$ where $p$ and $q$ are two primes that are different from each other, and let $m = p^2q$. Then $m$ divides $n^2 = p^2q^2$ but not $n = pq$. So we cannot use the same argument, with the same Lemma, to prove the Theorem for all $m$ that are not perfect squares. 

However, the Claim that if $m$ divides $n^2$ then $m$ divides $n$ \textit{should} hold true for all \textbf{prime} $m$. When $m = 3$ we had to consider two cases: where the remainder of dividing $n$ by $m$ was 1 or 2. In general there will be $m-1$ cases! We cannot go through them one by one (we don't know how many there are, since we don't know the value of $m$), so we will have to ``parametrize'' all the cases and handle them in a generic way.

\begin{lemma}
Assume $m$ and $n$ are positive integers and $m$ is prime. If $m$ divides $n^2$ then $m$ divides $n$.
\end{lemma}

\begin{proof}
1. Assume $m$ and $n$ are positive integers, $m$ is prime, and $m$ divides $n^2$.

2. By definition of divisibility, there exists an integer $k$ such that $mk = n^2$. (We notice that $k$ must be positive.)

3. By the Quotient-Remainder theorem there exist integers $q, r$ such that $n = qm + r$ where $0 \leq r < m$.

4. If $r = 0$ then $n = qm$ so $m$ divides $n$, and we are done. So now consider the case $r > 0$.

5. Then $n^2 = (qm + r)^2 = q^2m^2 + 2qmr + r^2$.

6. By (2) and (4) we have $q^2m^2 + 2qmr + r^2 = mk$.

7. Dividing by $m$ we get $q^2m + 2qr + \frac{r^2}{m} = k$.

8. Moving terms, we get $q^2m + 2qr - k = -\frac{r^2}{m}$.

9. Since $m$ is prime and $0 < r < m$, $r^2$ is not divisible by $m$. \textbf{(We need to prove this!)}

10. So the LHS of (8) is an integer, while the RHS of (8) is not an integer (because $r \neq 0$), a contradiction.

11. Our initial assumption must have been false, therefore $m$ divides $n$.
\end{proof}

Let's prove step (9). We have to use the Fundamental Theorem of Arithmetic and properties of prime numbers.

\begin{claim}
Assume $m$ is prime and $0 < r < m$ is an integer. Then $m$ does not divide $r^2$.
\end{claim}

\begin{proof}
1. Assume $m$ is prime and $0 < r < m$ is an integer.

2. By the Fundamental Theorem of Arithmetic, there exist primes $p_1, \ldots, p_n$ and positive integers $a_1, \ldots, a_n$ such that
$$
r = p_1^{a_1} \cdot p_2^{a_2} \cdot \ldots \cdot p_n^{a_n}
$$

3. By (2), we have $p_i \leq r$ for all $i = 1, \ldots, n$.

4. Since $0 < r < m$, by (3) we have $p_i < m$ for all $i = 1, \ldots, n$.

5. Since $m$ is prime, by (4) we have $m \nmid p_i$ for all $i = 1, \ldots, n$. 

6. Since $m$ is prime, by (5) $m$ does not divide any product of the primes $p_1, \ldots, p_n$ either.

7. By (2) we have 
$$
r^2 = p_1^{2a_1} \cdot p_2^{2a_2} \cdot \ldots \cdot p_n^{2a_n}
$$
so $r^2$ is a product of the primes $p_1, \ldots, p_n$.

8. By (7) and (6) $m$ does not divide $r^2$.
\end{proof}

With Lemma 2, we are able to generalize the Theorem to square roots of any primes (just repeat the proof for $\sqrt{3}$ where $m$ replaces $3$, and use Lemma 2 in the place of Lemma 1):

\begin{thm}
Assume $m$ is prime. Then $\sqrt{m}$ is irrational.
\end{thm}

Earlier we said that the theorem should hold not just for prime $m$, but any $m$ that is not a perfect square itself. However proving this greater generalization would require more work.

\section{Problem 3}

If we raise an irrational number to an irrational power, can the result be rational? Show that it can, by considering $\sqrt{2}^{\sqrt{2}}$ and arguing by cases.

\begin{proof}
1. \textbf{Case 1.} $\sqrt{2}^{\sqrt{2}}$ is rational.

1.1. We know that $\sqrt{2}$ is irrational (earlier Theorem from the lecture). 

1.2. So in this case, an irrational, namely $\sqrt{2}$, raised to an irrational power, namely $\sqrt{2}$, gives us a rational number, namely $\sqrt{2}^{\sqrt{2}}$. Therefore we proved the claim in this case.

2. \textbf{Case 2.} $\sqrt{2}^{\sqrt{2}}$ is irrational.

2.1. By the law of exponents $(a^b)^c = a^{bc}$ we have:
$$
\left(\sqrt{2}^{\sqrt{2}}\right)^{\sqrt{2}} = \sqrt{2}^{\sqrt{2} \cdot \sqrt{2}} = \sqrt{2}^2 = 2
$$

2.2. So, in this case, once again we have an irrational, namely $\sqrt{2}^{\sqrt{2}}$, raised to an irrational power, namely $\sqrt{2}$, that results in a rational number, namely 2. So we proved the claim in this case too.
\end{proof}

\section{Problem 4}

The fact that that there are irrational numbers $a, b$ such that $a^b$ is rational was proved earlier by cases. Unfortunately, that proof was {\it nonconstructive}: it didn’t reveal a specific pair, $a, b$ with this property. But in fact, it’s easy to do this: let $a \Coloneqq \sqrt{2}$ and $b \Coloneqq 2 \log_2(3)$. We know $a$ is irrational, and $a^b = 3$ by definition. Finish the proof that these values for $a, b$ work by showing that $2 \log_2(3)$ is irrational.

\begin{proof}
1. Argue by contradiction and assume $2\log_2(3)$ is rational.

2. By the definition of a rational number, there exist integers $n$ and $d$ such that $2\log_2(3) = \frac{n}{d}$, where $n$ and $d$ have no common divisors greater than 1. 

Without loss of generality we may assume $d > 0$.

3. Dividing both sides by 2, we get $\displaystyle \log_2(3) = \frac{n}{2d}$.

4. Using exponentiation with base 2 for both sides, we get  $2^{\log_2(3)} = 2^{n/2d}$.

5. By the definition of $\log_2$, we get $3 = 2^{n/2d}$.

6. Raising both sides to the power $2d$ we get $3^{2d} = 2^n$.

7. Dividing, we get
$$
\frac{3^{2d}}{2^n} = 1 
$$

8. Since $2$ and $3$ are different primes, $2^n$ cannot divide $3^{2d}$, unless $n = 0$. So by (7) we have $n = 0$.

9. By (8) and (2) we have $2\log_2(3) = \frac{0}{d} = 0$ which is a contradiction. (Because for the $\log_2$ function, the only root is $x = 1$. So $\log_2(3) \neq 0$.)

10. Therefore $2\log_2(3)$ is irrational.
\end{proof}
\end{document}